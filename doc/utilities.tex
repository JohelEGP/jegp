%!TEX root = std.tex
\rSec0[utilities]{General utilities library}

\rSec1[utilities.general]{General}

\pnum
This clause describes generally useful utilities.
These utilities are summarized in \tref{utilities.summary}.

\begin{libsumtab}{General utilities library summary}{utilities.summary}
\ref{utility}       & Utility components        & \tcode{<jegp/utility.hpp>}    \\
\end{libsumtab}

\rSec1[utility]{Utility components}

\rSec2[utility.syn]{Header \tcode{<jegp/utility.hpp>} synopsis}

\pnum
The header \libheaderdef{jegp/utility.hpp}
contains some basic constructs.

\indexheader{jegp/utility.hpp}%
\begin{codeblock}
namespace jegp
{
// \ref{utility.underlying}, \tcode{to_underlying}
template <class Enum>
constexpr std::underlying_type_t<Enum> to_underlying(Enum e) noexcept;

// \ref{static.downcast}, \tcode{static_downcast}
template <class DerivedRef, class Base>
constexpr DerivedRef static_downcast(Base&& base) noexcept;

// \ref{hash.combine}, \tcode{hash_combine}
template <class... Args>
constexpr std::size_t hash_combine(const Args&... args) noexcept(@\seebelow@);

} // namespace jegp
\end{codeblock}

\rSec2[utility.underlying]{\tcode{to_underlying}}

\indexlibraryglobal{to_underlying}%
\begin{itemdecl}
template <class Enum>
constexpr std::underlying_type_t<Enum> to_underlying(Enum e) noexcept;
\end{itemdecl}

\begin{itemdescr}
\pnum
\constraints
\tcode{std::is_enum_v<Enum>} is \tcode{true}.

\pnum
\returns
\tcode{static_cast<std::underlying_type_t<Enum>>(e)}.
\end{itemdescr}

\rSec2[static.downcast]{\tcode{static_downcast}}

\pnum
A \tcode{static_cast} that performs a downcast.

\indexlibraryglobal{static_downcast}%
\begin{itemdecl}
template <class DerivedRef, class Base>
constexpr DerivedRef static_downcast(Base&& base) noexcept;
\end{itemdecl}

\begin{itemdescr}
\pnum
Let \tcode{\placeholder{derived-ref}} be
\tcode{static_cast<DerivedRef>(std::forward<Base>(base))}.

\pnum
\constraints
\begin{itemize}
\item
  \tcode{DerivedRef} is a reference type,
\item
  \tcode{std::remove_cvref_t<DerivedRef>} is derived from
  \tcode{std::remove_cvref_t<Base>}, and
\item
  \tcode{\placeholder{derived-ref}} is well-formed.
\end{itemize}

\pnum
\expects
\tcode{\placeholder{derived-ref}} has well-defined behavior.

\pnum
\returns
\tcode{\placeholder{derived-ref}}.
\end{itemdescr}

\rSec2[hash.combine]{\tcode{hash_combine}}

\pnum
Insipred by \tcode{Boost.ContainerHash}.
Useful in the specializations of \tcode{std::hash}
whose \tcode{Key}'s salient parts consist of two or more objects.

\indexlibraryglobal{hash_combine}%
\begin{itemdecl}
template <class... Args>
constexpr std::size_t hash_combine(const Args&... args) noexcept(@\seebelow@);
\end{itemdecl}

\begin{itemdescr}
\pnum
\constraints
\begin{itemize}
\item
  $\tcode{sizeof...(Args)} \geq 2$ and
\item
  \tcode{std::hash<T>} is enabled (C++ Standard's [unord.hash])
  for all \tcode{T} in \tcode{Args}.
\end{itemize}

\pnum
\effects
Equivalent to:
\begin{codeblock}
std::size_t seed{0};
return (..., (seed ^= std::hash<Args>{}(args) + (seed << 6) + (seed >> 2)));
\end{codeblock}

\pnum
\remarks
The expression inside \tcode{noexcept} is equivalent to
\begin{codeblock}
(noexcept(std::hash<Args>{}(args)) && ...)
\end{codeblock}
\end{itemdescr}
